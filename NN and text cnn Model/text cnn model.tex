\documentclass{standalone}
\usepackage{tikz}

\begin{document}
	\pagestyle{empty}
	
	\def\layersep{2.5cm}
	
\begin{tikzpicture}[shorten >=1pt,->,draw=black!50, node distance=\layersep]
\tikzstyle{every pin edge}=[<-,shorten <=1pt]
\tikzstyle{neuron}=[circle,fill=black!25,minimum size=17pt,inner sep=0pt]
\tikzstyle{Embedding layer}=[neuron, fill=green!50];
\tikzstyle{output neuron}=[neuron, fill=red!50];
\tikzstyle{X neuron}=[neuron, fill=yellow!50];
\tikzstyle{hidden neuron}=[neuron, fill=blue!50];
\tikzstyle{annot} = [text width=4em, text centered]

% Draw the input layer nodes
\node[Embedding layer, pin=left:OccupationID] (I-1) at (0,-1) {};
\node[Embedding layer, pin=left:Gender] (I-2) at (0,-2) {};
\node[Embedding layer, pin=left:Age] (I-3) at (0,-3) {};
\node[Embedding layer, pin=left:UserID] (I-4) at (0,-4) {};
\node[Embedding layer, pin=left:MovieID] (I-6) at (0,-6) {};
\node[Embedding layer, pin=left:Genres] (I-7) at (0,-7) {};
\node[Embedding layer, pin=left:Title] (I-9) at (0,-9) {Text CNN};

% Draw the hidden layer nodes
\foreach \name / \y in {1,...,4,6,7,9}
\path[yshift=0 cm]
node[hidden neuron] (H-\name) at (\layersep,-\y cm) {32 $\times$1};

% Draw the output layer node
\node[output neuron, right of=H-3] (O1) {256$\times$1};
\node[output neuron, right of=H-6] (O2) {256$\times$1};
% Connect every node in the input layer with every node in the
% hidden layer.
\foreach \source in {1,2,3,4}
\foreach \dest in {1,2,3,4}
\path (I-\source) edge (H-\dest);

\foreach \source in {6,7,9}
\foreach \dest in {6,7,9}
\path (I-\source) edge (H-\dest);

% Connect every node in the hidden layer with the output layer
\foreach \source in {1,...,4}
\path (H-\source) edge (O1);

\foreach \source in {6,7,9}
\path (H-\source) edge (O2);

\node[X neuron, right of=O1  ] (O3) {Rating};
\path (O1) edge (O3);
\path (O2) edge (O3);
% Annotate the layers
\node[annot,above of=H-1, node distance=1cm] (hl) {Hidden layer};
\node[annot,above of=I-1, node distance=1cm] (i1) {Embedding layer};
\node[annot,left of=il] {Input layer};
\node[annot,right of=hl] {Feature layer};
\node[annot,above of=O3, node distance=3cm] ( ) {Output layer};
\end{tikzpicture}	
	
	
	





\end{document}